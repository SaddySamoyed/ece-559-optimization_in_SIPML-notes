\usepackage{graphicx,amsmath,amsfonts,amscd,amssymb,bm,bbm,url,color,latexsym}
%\usepackage[utf8]{inputenc} % allow utf-8 input
\usepackage[T1]{fontenc}    % use 8-bit T1 fonts
\usepackage{hyperref}       % hyperlinks
\usepackage{algorithm,algorithmicx}
\usepackage{algpseudocode}
\usepackage{booktabs}       % professional-quality tables
\usepackage{nicefrac}       % compact symbols for 1/2, etc.
\usepackage{microtype}      % microtypography
\usepackage{multirow}
\usepackage{subcaption}
\usepackage{smartdiagram}
\usepackage[cal=rsfso,calscaled=.96]{mathalfa}
\usepackage{mwe}
\usepackage[font={footnotesize,sf},labelfont=bf]{caption}
\usepackage[nameinlink]{cleveref}
\setlength{\captionmargin}{30pt}
\usepackage{verbatim}
\usepackage{framed}
\usepackage{comment}
\usepackage{wrapfig}
\usepackage{tikz}
\usepackage{tikz-3dplot}
\usepackage{pgfplots}
%\usepackage{subfig}
\usepackage[toc, page]{appendix}
\usepackage{tablefootnote}
\usepackage{tcolorbox}
%\usepackage{pifont}
\usepackage{authblk}
\usepackage{fancyhdr}
%\usepackage[round,compress]{natbib}
%\usepackage[usenames,dvipsnames]{color}
%\usepackage{harvard}
\usepackage{array} % This is for the DMP
%\usepackage[left=1.0in,right=1.0in,top=1.0in,bottom=1.0in,margin=1in]{geometry}
%\usepackage[pagebackref=false,breaklinks=true,colorlinks,letterpaper=true,bookmarks=false,citecolor=OliveGreen]{hyperref}


%\pagestyle{empty}
%\pagestyle{plain}
%\parindent = 2 pt
\parskip = 1.6 pt
\setlength{\topsep}{0.0in}
\setlength{\topskip}{0.0in}

% pagebreak for long equations
%\allowdisplaybreaks

\hypersetup{
    colorlinks=true,%
    citecolor=blue,%
    filecolor=blue,%
    linkcolor=blue,%
    urlcolor=blue
}

%\usepgfplotslibrary{ternary}
%\newcommand{\edited}[1]{{\color{blue}{#1}}}
%\numberwithin{equation}{section}

%\def \endprf{\hfill {\vrule height6pt width6pt depth0pt}\medskip}
\newenvironment{proof}{\noindent {\bf Proof} }{\endprf\par}

\setlength{\abovedisplayskip}{1pt}
\setlength{\belowdisplayskip}{1pt}
\setlength{\belowcaptionskip}{-11pt}

\providecommand{\keywords}[1]{\textbf{\textit{Index terms---}} #1}

%%%%%%%%%%%%%%
\renewcommand\floatpagefraction{.95}
\renewcommand\topfraction{.95}
\renewcommand\bottomfraction{.95}
\renewcommand\textfraction{.05}
\setcounter{totalnumber}{50}
\setcounter{topnumber}{50}
\setcounter{bottomnumber}{50}
%%%%%%%%%%%%%%
%--------------
\newtheorem{theorem}{Theorem}[section]
\newtheorem{lemma}[theorem]{Lemma}
\newtheorem{corollary}[theorem]{Corollary}
\newtheorem{proposition}[theorem]{Proposition}
\newtheorem{definition}[theorem]{Definition}
\newtheorem{conjecture}[theorem]{Conjecture}
\newtheorem{problem}[theorem]{Problem}
\newtheorem{claim}[theorem]{Claim}
\newtheorem{remark}[theorem]{Remark}
\newtheorem{example}[theorem]{Example}
\newtheorem{assumption}[theorem]{Assumption}

\newcommand{\eng}[1]{\latintext#1\greektext}
\newcommand{\red}[1]{\textcolor{red}{#1}}
\newcommand{\blue}[1]{\textcolor{blue}{#1}}
\newcommand{\indep}{\rotatebox[origin=c]{90}{$\models$}}

\newcommand{\conv}{\circledast}
\newcommand{\mb}{\mathbf}
\newcommand{\mc}{\mathcal}
\newcommand{\mf}{\mathfrak}
\newcommand{\md}{\mathds}
\newcommand{\bb}{\mathbb}
\newcommand{\magnitude}[1]{ \left| #1 \right| }
\newcommand{\set}[1]{\left\{ #1 \right\}}
\newcommand{\condset}[2]{ \left\{ #1 \;\middle|\; #2 \right\} }
\newcommand{\mr}{\mathrm}
\newcommand{\inprod}[2]{\langle#1,#2\rangle}
\newcommand{\parans}[1]{\left(#1\right)}

\newcommand{\reals}{\bb R}
\newcommand{\proj}{\mathrm{proj}}
\newcommand{\E}{\mathbb{E}}

\newcommand{\eps}{\varepsilon}
\newcommand{\R}{\reals}
\newcommand{\Cp}{\bb C}
\newcommand{\Z}{\bb Z}
\newcommand{\N}{\bb N}
\newcommand{\Sp}{\bb S}
\newcommand{\Ba}{\bb B}
\newcommand{\indicator}[1]{\mathbbm 1_{#1}}
\newcommand{\Brac}[1]{\left\{ #1 \right\}}
\newcommand{\brac}[1]{\left[ #1 \right]}
\newcommand{\paren}[1]{\left( #1 \right)}

\newcommand{\col}{\mathrm{col}}

\newcommand{\wh}{\widehat}
\newcommand{\wt}{\widetilde}
\newcommand{\ol}{\overline}
\newcommand{\wc}{\widecheck}

\newcommand{\norm}[2]{\left\| #1 \right\|_{#2}}
\newcommand{\abs}[1]{\left| #1 \right|}
\newcommand{\innerprod}[2]{\left\langle #1,  #2 \right\rangle}
\newcommand{\prob}[1]{\bb P\left[ #1 \right]}
\newcommand{\expect}[1]{\bb E\left[ #1 \right]}
\newcommand{\function}[2]{#1 \left(#2\right)}
\newcommand{\integral}[4]{\int_{#1}^{#2}\; #3\; #4}
\newcommand{ \init }[1]{{\mathbf #1}_{\mathrm{init}}}

\newcommand{\ie}{i.e.,\ }
\newcommand{\eg}{e.g.,\ }
\newcommand{\etal}{et al.\ }


\newcommand{\qed}{{\unskip\nobreak\hfil\penalty50\hskip2em\vadjust{}
           \nobreak\hfil$\Box$\parfillskip=0pt\finalhyphendemerits=0\par}}

\newcommand{\res}[2]{\mb \iota_{ #1 \rightarrow #2  }}
\newcommand{\shift}[2]{{\mathrm{s}_{#2}}\left[#1\right]}
\newcommand{\Shift}[2]{{\wh{\mathrm{s}}_{#2}}\left[#1\right]}

\renewcommand{\P}{\mathbb{P}}
\renewcommand{\arraystretch}{1}
\renewcommand{\baselinestretch}{1.020}
\renewcommand{\mathbf}{\boldsymbol}

% to declare new operator
% \DeclareMathOperator{\xxx}{xxx}
\DeclareMathOperator{\dist}{dist}
\DeclareMathOperator{\poly}{poly}
\DeclareMathOperator{\rank}{rank}
\DeclareMathOperator{\trace}{tr}
\DeclareMathOperator{\supp}{supp}
\DeclareMathOperator{\vect}{vec}
\DeclareMathOperator{\diag}{diag}
\DeclareMathOperator{\sign}{sign}
\DeclareMathOperator{\grad}{grad}
\DeclareMathOperator{\Hess}{Hess}
\DeclareMathOperator{\mini}{minimize}
\DeclareMathOperator{\maxi}{maximize}
\DeclareMathOperator{\st}{subject\; to}

%% Other definitions

%\newcommand{\qq}[1]{{\color{blue}{Qing: #1}}}
\newcommand{\sz}[1]{{\color{blue}{\bf Simon: #1}}}
\newcommand{\zz}[1]{{\color{purple}{Zhihui: #1}}}
\newcommand{\yz}[1]{{\color{blue}{\bf Yuqian: #1}}}
\newcommand{\td}[1]{{\color{magenta}{\bf TODO: #1}}}
\newcommand{\edit}[1]{{\color{black}{#1}}}
\newcommand{\edited}[1]{{\color{black}{#1}}}

\newcommand{\forceindent}{\leavevmode{\parindent=2em\indent}}

%\titlespacing*{\subsubsection}{0pt}{*1.5}{*0}
%\titlespacing*{\paragraph}{11pt}{*1}{*1}
%\titleformat*{\paragraph}{\itshape}
\newcommand{\myparagraph}[1]{\medskip\noindent\textbf{#1.}}
%\newcommand{\myparagraph}[1]{\smallskip\noindent\textbf{#1.}}
%\newcommand{\mysubparagraph}[1]{\smallskip\noindent\emph{#1:}}


% this handles hanging indents for publications
\def\rrr#1\\{\par
\medskip\hbox{\vbox{\parindent=2em\hsize=6.12in
\hangindent=4em\hangafter=1#1}}}

%\graphicspath{{../figs/}}
%\graphicspath{{figs/}}
% Data Management Stuff
% Counter for Data Management Plan "products"
\newcounter{productC}
\newcommand{\rproduct}{%
        \stepcounter{productC}%
        \theproductC}

%%%%%%%%%%%%%%%%%%%%%%%%%%%%%%%%%%%%
% FORMATTING
%%%%%%%%%%%%%%%%%%%%%%%%%%%%%%%%%%%%

%\usepackage[compact,small]{titlesec}
%\titlespacing*{\subsection}{0pt}{*1.5}{*0.5}

%\titleformat*{\paragraph}{\itshape}
\usepackage{enumitem}
%\setlist{topsep=0.1cm,itemsep=-0.1cm,itemindent=0.0cm,leftmargin=0.6cm}

\usepackage{ragged2e}
\newcolumntype{P}[1]{>{\RaggedRight\hspace{0pt}}p{#1}}

%\renewcommand{\thesection}{\Alph{section}}   % set the section counter to Alpha
%\renewcommand{\thepage}{\thesection-{\arabic{page}}}
\newcommand{\margin}[1]{\marginpar{\color{red}\ttfamily\footnotesize#1}}













